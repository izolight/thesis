\chapter{Use-Cases}
\label{ch:usecases}

\section{Document Signing}

\subsection{Interactive Qualified Signatures}
It should be possible to let the signature service sign documents with a qualified signature.
In this case the person who wishes to sign the documents needs to authenticate with the IDP for each document that he wishes to have signed.

\subsection{Bulk Advanced Signatures}
It should also be possible to have a longer running session for bulk document signing.
In this case the authentication will be cached for a certain duration without needing to re-authenticate for each document.
In this mode only advanced signatures can be created.

\section{Signature Validation}

\subsection{Offline Validation}
The signatures can be verified offline with just the document, the signature and the verification program.
This mode will only be supported on desktops(Linux, Windows, MacOS X).

\subsection{Semi-Online Validation}
For mobile clients a website will be provided to validate the signature by providing the document and the signature.
The validation will happen in the browser, but no files will be uploaded in the process.