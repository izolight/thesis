\chapter{Introduction}
\label{sec:introduction}

Today, we use a number of computing devices interchangeably on a daily basis: a desktop workstation at the office,
a laptop computer on the move, a tablet in the living room and of course, always by our side, the smartphone.
In an increasingly cloudified and mobile world our expectation is to be able to do our work all the same,
regardless of the computing device we use, or where we are.

We start editing a text document in Google Docs on our desktop workstation at the office,
work on it a bit more on our laptop while travelling by train,
and proofread it later on the smartphone.
This device- and location-independent way of working has become the standard in recent years,
and users have started to expect it from their IT devices.

It's difficult to meet this expectation with the way electronic signatures are usually created today,
using certificates stored on on smartcards,
plugged into a laptop,
using a specialised card reader and accompanying software.
It's annoying and inconvenient having to carry around cables and adaptors, and a lot can go wrong:
a random operating system update breaking driver compatibility with the card reader, for example,
leaving us dead in the water.
If we want to make this easier on the user and to drive usage of electronic signatures and even make them mainstream,
we have to do better.

At the root of this inconvenience is the requirement that the user keep their private key physically with them,
stored in a manner making it difficult for anyone to steal it: on a smartcard.
Any IT professional knows full well this demand isn't made from users in order to annoy them but because it is
- more or less - the only practical \textit{and} secure way to have users store their private key.

So-called Remote Signing Services aim to eliminate the need for people to carry their private key with them,
and to locally create signatures,
in the hope for improved ease of use, and eventually, greater adoption of digitally signing documents.
However, allowing someone (the signing service, in this case) to be able to sign documents in place of the user introduces a number of serious security and confidentiality problems.


In this thesis, we analyse and address these problems,
and we implement the proposed solutions in a fully functional Remote Signing Service,
thereby showing that they work in the real world and not just on paper.

We will allow people to create electronic signatures,
no matter where they are, or what device they're using,
in a secure manner.
Building on our previous work of Project 2~\cite{projekt2}, we show how it is possible to securely integrate \gls{OIDC} authentication with remote digital signatures.
We expand upon this previous work and show how it is possible to have a remote signing service with the capability of signing on the users' behalf without the need for completely trusting that service.
Furthermore, we compare our solutions to those proposed by an industrial consortium led by Adobe Inc.,
and we show in which ways we believe our approach to be superior.

\subsection{Purpose of this document}\label{subsec:purpose-of-this-document}

In this specification document we will outline the objectives, scope and methodologies for our thesis as well as provide a project timeline.
