\chapter*{Functional Requirements}
\label{ch:functionalrequirements}

\section{Terminology}

As always, the usual modal verbs are to be interpreted as in RFC 2119~\cite{rfc2119}.

\subsection{Practically Impossible}
"Practically impossible" means the probability of it being possible is not zero, but so small for it not to matter in practice.
An example for this would be finding the prime factors of the product of two carefully chosen 1024 bit numbers within 24 hours.

\subsection{Being made difficult}
"Made difficult" means something far from impossible for someone with near-unlimited resources like a state actor,
but extremely difficult if not impossible even for a highly skilled single person with the resources expected for a single person.
For example, stealing a smart card from someone and misusing the contained private key.

\section{Signature Requirements}
\label{sec:signaturerequirements}
\subsection{Authenticity}
It must be practically impossible for anyone to forge a signature without it being detected upon signature verification.

\subsection{Integrity}
It must be practically impossible for anyone to modify a signed document without being detected upon signature verification.
A secure hash algorithm must be used for hashing the document.
Secure means the algorithm to be pre-image as well as collision-resistant as validated by the \gls{NIST} Cryptographic Algorithm Validation Program~\cite{nistcavp}.

\subsection{Verifiability}
Anyone must be able to verify the authenticity of a signature and the integrity of the signed document.

\subsection{Non-repudiation}
It must be practically impossible for anyone to deny having signed a document.

\subsection{Long-Term Validation}
Signatures must be suitable for \gls{LTV} using an RFC3161~\cite{rfc3161} timestamp.

\subsection{Secure Coupling of Authentication and Signature}
It must be practically impossible for anyone to abuse a stolen \gls{OIDC} ID token to sign a document other than intended by The Signer.

\subsection{Authentication Protocol}
Standard \gls{OIDC} must be used for authenticating The Signer as specified in the standard~\cite{oidc}.

\subsection{Supported File Formats}
It must be possible to sign any file, regardless of its format.

\subsection{Bulk Signatures}
For qualified signatures, it must be possible to sign more than one document at once.

For advanced signatures, it may be possible to sign several documents one after the other without requiring re-authentication.

\section{Signature Server Requiremenents}
\label{sec:signatureserverrequirements}
\subsection{CA Key Security}
Technical measures must be taken to make it difficult for the private keys of the signing \gls{CA} to be stolen.

\subsection{User Key Security}
Technical measures must be taken to make it difficult to steal the private keys generated on behalf of the users.

\subsection{No unauthorised identity delegation}
It must be practically impossible for the signing server to create a signature on its own.

\subsection{Random Number Generation}
The \gls{RNG} used for generating signatures must be a cryptographically secure.

\subsection{REST API}
The Signature Server must offer a \gls{REST} \gls{API} that can be used by third parties to interface with the signing service,
for example in order to implement custom frontends or to include it as part of their product,
or for users that don't like \gls{GUI}s.

\chapter*{Non-Functional Requirements}
\label{ch:nonfunctionalrequirements}

\subsection{Efficient Signature File Format}
The file format for the signature file shall be based on our previous work~\cite{projekt2}.

\subsection{Protection of Information}
Information not strictly required by the party in order to fulfil their function must not be disclosed to the aforementioned party.
In particular, the document to be signed must not be disclosed to the signing server nor to the \gls{IDP}.

\subsection{Offline Validation}
The Verifier must be enabled to verify signatures without an active internet connection using a desktop or laptop computer running GNU/Linux, MacOS or Windows.

\section{IDP Requirements}
Anything related to the \gls{IDP} is out of scope for our thesis, except for specifying what we require of the same.
We assume to be using an existing, \gls{OIDC}-conforming \gls{IDP} providing the required registration and authentication levels.

\subsection{Support for OIDC}
The \gls{IDP} must support standard \gls{OIDC} as specified in the standard~\cite{oidc}.


\subsection{Levels of Assurance}
The \gls{IDP} must support \gls{AAL} 2 authentication for advanced signatures, and \gls{AAL} 3 authentication for qualified signatures as specified in the \gls{NIST} publication~\cite{nistdigitalidentityguidelines}.

