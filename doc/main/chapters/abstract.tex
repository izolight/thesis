\chapter*{Abstract}\label{ch:abstract}
To enable a future digital single market,
the European Union has proposed a standard for Remote Digital Signing.
Remote Signing means that cryptographic operations are outsourced to a trust service provider,
including the storage of the private keys.
This is done to make digital signing more user-friendly,
and to remove the burden of key management from the end-user.
However, this means that effectively, end users have no control over their private keys.
Remote signing services implementing the EU standard could sign arbitrary documents in their users' names,
without their authorisation, and without them even noticing.

In this thesis,
we expand upon the EU's standard by designing a remote signing service in such a way that cannot sign any document without the users' direct authorisation,
despite having control over the users' private keys.
Contrary to proposals involving secret sharing schemes,
with our design, no client-side software is necessary, except for a web browser.

To illustrate our solution,
we create a proof-of-concept implementation consisting of the signing service itself as well as a
 platform-independent verification program capable of both online and offline verification.




