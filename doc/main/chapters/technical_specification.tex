\chapter{Technical Specification}
\section{REST API}
TODO
\section{Signature File Format}
TODO
\subsection{Long-Term Validation}
\gls{LTV} allows for the validation of signatures long after the document was signed~\cite{etsipades}.

We need \gls{LTV} for two main reasons:
\begin{enumerate}
    \item Imagine if the \gls{CA} were revoked that was used for the signatures: all signatures created using the same \gls{CA} would become invalid instantly, making countless documents, constracts and the like unverifiable.
    \item Extending the validity of the signature beyond the lifetime of the \gls{CA} used to sign it, for signatures that need to remain valid and verifiable for a very long time.
\end{enumerate}
In order for us to achieve this, all required elements for signature validation must be embedded into the signature file.
Without the addition of these elements, a signature can only be validated for a limited time.
This limitation occurs because the \gls{CA}s eventually expire, or get revoked.
Once the \gls{CA} certificate has expired, the issuing authority is no longer responsible for providing the revocation status information on that certificate.
Without the confirmed revocation status information on the signing keys, the signature cannot be validated.

To overcome this limitation, the following information has to be embedded into the signature:
\begin{enumerate}
    \item A timestamp on the signature
    \item The signing certificate
    \item The certificates used and their revocation status (\gls{OCSP} and \gls{CRL})
    \item An archive timestamp of the previous content
\end{enumerate}

The archive timestamp establishes the date in which the information collected was issued.
Provided the archive timestamp is valid,
we can be sure that the revocation information was issued at that time,
and check the validity of the signing certificate and the \gls{CA} certificate chain.
Thus we can be certain that it was not revoked at the point in time the document was signed.
This allows us to extend the validity of the signature past the expiration time of the \gls{CA}.

However, this does not extend the validity of the signature indefinitely,
it merely extends the expiration until the expiration time of the timestamping authority's certificate.

For many cases this may be enough, but it doesn't quite yet allow for long-time archival.
When the timestamping certificate's expiration is impending,
the signature expiration time has to be extended by adding another timestamp signed by a \gls{CA} not yet close to expiration.
This re-stamping has to be repeated periodically in order to keep the signature valid and verifiable.
This allows for near-indefinite archival.

