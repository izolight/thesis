\chapter{Introduction}
\label{ch:Introduction}

The trend for Signing Service Solutions goes in the direction of Remote Signing.
Users no longer hold on to the cryptographic keys themselves,
instead a remote service stores the users' signing keys or even generates the keys on-demand.
This has a serious drawback:
such a signing service is able to sign without user's knowledge or consent,
thus it must offer complete trustworthiness.
In this thesis,
we provide a specification as well as a proof-of-concept where the signing process is tied to the identity of the user,
such that the signing service by itself cannot sign a document.

The implementation consists of the remote signing service itself exposing a \gls{REST} \gls{API},
a cross-platform verification program offering both online and offline verification,
and a cross-platform frontend authenticating the users through a trusted \gls{OIDC} \gls{IDP}.


\section{Previous Works}
\label{sec:previousworks}

We build upon our previous work of Project 2~\cite{projekt2},
where we specified the authentication process for qualified signatures,
non-qualified batch signatures,
a signature file format,
as well as - to our knowledge - pioneering the secure integration of a digital signature with an \gls{OIDC} ID token without requiring any change to the \gls{IDP}.

In this thesis, we expand upon our previous work substantially both functionally and conceptually.
This thesis is not just an implementation of an existing concept.

\section{Overview of Contents}\label{sec:overview}
TODO

